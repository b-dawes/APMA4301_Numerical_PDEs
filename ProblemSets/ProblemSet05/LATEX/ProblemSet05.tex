\documentclass{article}

\usepackage{amsmath}
\usepackage{graphicx}

\begin{document}
\title{APMA 4301: Problem Set 5}
\author{Brian Dawes}
\maketitle

\section*{1.}
\subsection*{(a) Butcher Tableaus}
\renewcommand{\arraystretch}{1.2}
\subsubsection*{Forward Euler}
\begin{equation}
\begin{array}{c|cc}
0 & 0 \\
1 & 1 & 0 \\
\hline
 & 0 & 1
\end{array}
\end{equation}

\subsubsection*{Backward Euler}
\begin{equation}
\begin{array}{c|cc}
0 & 0 \\
1 & 0 & 1 \\
\hline
& 0 & 1
\end{array}
\end{equation}

\subsubsection*{Mid-Point}
\begin{equation}
\begin{array}{c|cc}
0 & 0 \\
\frac{1}{2} & \frac{1}{2} & 0 \\
\hline
& 0 & 1
\end{array}
\end{equation}


\subsubsection*{Improved Euler (RK2)}
\begin{equation}
\begin{array}{c|cc}
0 & 0 \\
1 & 1 & 0 \\
\hline
& \frac{1}{2} & \frac{1}{2}
\end{array}
\end{equation}


\subsubsection*{Trapezoidal}
\begin{equation}
\begin{array}{c|cc}
0 & 0 \\
1 & 0 & 1 \\
\hline
& \frac{1}{2} & \frac{1}{2}
\end{array}
\end{equation}

\subsubsection*{Classical 4th order Runge-Kutta (RK4)}
\begin{equation}
\begin{array}{c|cccc}
0 & 0 \\
\frac{1}{2} & \frac{1}{2} & 0 \\
\frac{1}{2} & 0 & \frac{1}{2} & 0 \\
1 & 0 & 0 & 1 & 0 \\
\hline
& \frac{1}{6} & \frac{1}{3} & \frac{1}{3} & \frac{1}{6}
\end{array}
\end{equation}


\subsubsection*{TR-BDF2}

\subsection*{(b) Explicit Form}
\subsubsection*{Forward Euler}
\begin{equation}
u_1(k)=(1+z)u_0
\end{equation}

\subsubsection*{Backward Euler}
\begin{align}
u_1(k)&=u_0+zu_1(k) \nonumber\\
u_1(k)&=\frac{1}{1-z}u_0
\end{align}

\subsubsection*{Mid-Point}
\begin{equation}
u_1(k)=(1+z+z^2/2)u_0
\end{equation}

\subsubsection*{Improved Euler (RK2)}
\begin{equation}
u_1(k)=(1+z+z^2/2)u_0
\end{equation}

\subsubsection*{Trapezoidal}
\begin{align}
u_1(k)&=u_0+z/2(u_0+u_1(k)) \nonumber\\
u_1(k)&=\frac{1+z/2}{1-z/2}u_0
\end{align}

\subsubsection*{Classical 4th order Runge-Kutta (RK4)}
\begin{align}
u_1(k)&=(1/6+1/3+z/6+1/3+z/6+z^2/12+1/6+z/6+z^2/12+z^3/12)u_0 \nonumber\\
&=(1+z/2+z^2/6+z^3/12)u_0
\end{align}

\subsubsection*{TR-BDF2}

\subsection*{(c) Step Error}
\subsubsection*{Forward Euler}
\begin{equation}
e^z-R(z)=z^2/2+O(z^3)
\end{equation}

\subsubsection*{Backward Euler}
\begin{align}
e^z-R(z)&=(1+z+z^2/2+\dots)-(1+z+z^2+\dots) \nonumber\\
&=-z^2/2+O(z^3)
\end{align}

\subsubsection*{Mid-Point}
\begin{equation}
e^z-R(z)=z^3/6+O(z^4)
\end{equation}

\subsubsection*{Improved Euler (RK2)}
\begin{equation}
e^z-R(z)=z^3/6+O(z^4)
\end{equation}

\subsubsection*{Trapezoidal}
\begin{align}
e^z-R(z)&=(1+z+z^2/2+z^3/6\dots)-(1+z+z^2/2+z^3/4\dots) \nonumber\\
&=-z^3/12+O(z^4)
\end{align}

\subsubsection*{Classical 4th order Runge-Kutta (RK4)}
\begin{equation}
e^z-R(z)=z^5/5!+O(z^6)
\end{equation}

\subsubsection*{TR-BDF2}

\subsection*{(d) Stability Plots}
\subsubsection*{Forward Euler}

\subsubsection*{Backward Euler}

\subsubsection*{Mid-Point}

\subsubsection*{Improved Euler (RK2)}

\subsubsection*{Trapezoidal}

\subsubsection*{Classical 4th order Runge-Kutta (RK4)}

\subsubsection*{TR-BDF2}

\subsection*{(e) Stable Time Steps}
\subsubsection*{Forward Euler}
\begin{equation}
|1+\lambda k_{max}|=1\implies k_{max}=-\frac{2}{\lambda}
\end{equation}

\subsubsection*{Backward Euler}
\begin{equation}
|\frac{1}{1-\lambda k_{max}}|=1\implies k_{maz}=\infty
\end{equation}

\subsubsection*{Mid-Point}
\begin{equation}
|1+\lambda k_{max}+(\lambda k_{max})^2/2|=1\implies k_{max}=-\frac{1}{\lambda}
\end{equation}

\subsubsection*{Improved Euler (RK2)}
\begin{equation}
|1+\lambda k_{max}+(\lambda k_{max})^2/2|=1\implies k_{max}=-\frac{1}{\lambda}
\end{equation}

\subsubsection*{Trapezoidal}
\begin{equation}
|\frac{1+\lambda k_{max}/2}{1-\lambda k_{max}/2}|=1\implies k_{maz}=\infty
\end{equation}

\subsubsection*{Classical 4th order Runge-Kutta (RK4)}

\subsubsection*{TR-BDF2}

\section*{2.}
\begin{equation}
\frac{\partial T}{\partial t}=\nabla^2T;\quad T(\mathbf x,0)=A\exp\left[\frac{-\mathbf{x}^T\mathbf{x}}{\sigma^2}\right]
\end{equation}

\subsection*{(a)}
\begin{equation}
T(\mathbf{x},t)=\frac{A}{1+2t/\sigma^2}\exp\left[\frac{\mathbf{x}^T\mathbf x}{\sigma^2+4t}\right]
\end{equation}
This obviously satisfies the initial condition at $t=0$.

\begin{equation}
\frac{\partial T}{\partial t}=\frac{A}{1+2t/\sigma^2}\left[\frac{4\mathbf{x}^T\mathbf{x}}{(\sigma^2+4t)^2}-\frac{2}{\sigma^2+4t}\right]\exp\left[\frac{\mathbf{x}^T\mathbf x}{\sigma^2+4t}\right]=\nabla^2T
\end{equation}

\subsection*{(b)}
\subsubsection*{i.}
\begin{align}
F(u_i)&=\int_\Omega u_t[u_i-u_n+(1-\theta)\nabla^2 u_n+\theta\nabla u_i^2]\,d\mathbf x \nonumber\\
&=\int_\Omega u_t(u_i-u_n)+k\nabla u_t\cdot[(1-\theta)\nabla u_n+\theta\nabla u_i]\,d\mathbf x
\end{align}
Note that the surface terms are 0 due to the Neumann boundary condition. 

In UFL, this can be written as:
\begin{verbatim}
F = (k*inner(grad(u_t),theta*grad(u_i)
    +(1-theta)*grad(u_n))+inner(u_t,(u_i-u_n)))*dx
\end{verbatim}

\subsubsection*{ii.}
Differentiating F, we get the Jacobian:
\begin{equation}
J(u_i)=u_t-k\theta\nabla^2u_t
\end{equation}
after using the identity $\frac{\partial}{\partial u}(f\cdot\nabla u)=-\nabla f$.

In UFL this can be done implicitly with:
\begin{verbatim}
J = derivative(F,us_i,us_a)
\end{verbatim}

\subsubsection*{iii.}
The file \verb|diffusion.tfml| solves this problem with $\theta=0$. The file \verb|diffusion.shml| loops over the problem for $\theta=0,0.5,1$. The L1 error functional is included in the diagnostics.

\subsubsection*{iv.}

\subsubsection*{v.}
For $\theta=0.1$, we have an L1 error of 8.7694862558e-05 at $t=0.012$ and $k=0.0002$. For $\theta=0.5$, we have an L1 error of 3.3144604514e-05 at $t=0.012$ and $k=0.0002$. Since the step error of the backwards Euler method is second order, we should get the same error when running with $k=0.0000285$. Using this timestep, we get an error of 3.1004511669e-05, which is actually slightly better than the trapezoidal scheme.
\end{document}