\documentclass{article}

\usepackage{amsmath}
\usepackage{graphicx}

\begin{document}
\title{APMA 4301: Problem Set 2}
\author{Brian Dawes}
\date{\today}
\maketitle
\section*{Problem 1}
\subsection*{a)}
Using the Taylor expansion, we can write:
\begin{equation}
u(x) = u(\bar x)+(x-\bar x)u^\prime(\bar x)+\frac{(x-\bar x)^2}{2}u''(\bar x)+\frac{(x-\bar x)^3}{6}u^{(3)}(\bar x)
\end{equation}

Plugging in $x=0,h,2h$, we get:
\begin{eqnarray}
u_0 \approx u(\bar x)-\bar xu^\prime(\bar x)+\frac{\bar x^2}{2}u''(\bar x)-\frac{\bar x^3}{6}u^{(3)}(\bar x) \\
u_1 \approx u(\bar x)+(h-\bar x)u^\prime(\bar x)+\frac{(h-\bar x)^2}{2}u''(\bar x)+\frac{(h-\bar x)^3}{6}u^{(3)}(\bar x) \\
u_2 \approx u(\bar x)+(2h-\bar x)u^\prime(\bar x)+\frac{(2h-\bar x)^2}{2}u''(\bar x)+\frac{(2h-\bar x)^3}{6}u^{(3)}(\bar x)
\end{eqnarray}
where $u_0=u(0)$, $u_1=u(h)$, and $u_2=u(2h)$.

Now we want to find stencil weights $s_0,s_1,s_2$ such that:
\begin{equation}
s_0u_0+s_1u_1+s_2u_2=u'(\bar x)\ \mathrm{or}\ u''(\bar x)
\end{equation}

This can be represented in matrix form as:
\begin{equation}
\begin{bmatrix}
1 & 1 & 1 \\
-\bar x & h-\bar x & 2h-\bar x \\
\bar x^2/2 & (h-\bar x)^2/2 & (2h-\bar x)^2/2
\end{bmatrix}
\begin{bmatrix}
s_0 \\ s_1 \\ s_2
\end{bmatrix}
=
\begin{bmatrix}
0 \\ 1 \\ 0
\end{bmatrix}
\ \mathrm{or}\ 
\begin{bmatrix}
0 \\ 0 \\ 1
\end{bmatrix}
\end{equation}
which can be solved by diagonalizing the left side of:
\begin{equation}
\begin{bmatrix}
1 & 1 & 1 & 0 & 0\\
-\bar x & h-\bar x & 2h-\bar x & 1 & 0\\
\bar x^2/2 & (h-\bar x)^2/2 & (2h-\bar x)^2/2 & 0 & 1
\end{bmatrix}
\end{equation}

\subsubsection*{Case 1: $\bar x=0$}
Our matrix becomes:
\begin{equation}
\begin{bmatrix}
1 & 1 & 1 & 0 & 0\\
0 & h & 2h & 1 & 0\\
0 & h^2/2 & 2h^2 & 0 & 1
\end{bmatrix}
\end{equation}

\subsubsection*{Case 2: $\bar x=h/2$}
Our matrix becomes:
\begin{equation}
\begin{bmatrix}
1 & 1 & 1 & 0 & 0\\
-h/2 & h/2 & 3h/2 & 1 & 0\\
h^2/8 & h^2/8 & 9h^2/8 & 0 & 1
\end{bmatrix}
\end{equation}

\subsubsection*{Case 3: $\bar x=h$}
Our matrix becomes:
\begin{equation}
\begin{bmatrix}
1 & 1 & 1 & 0 & 0\\
-h & 0 & h & 1 & 0\\
h^2/2 & 0 & h^2 & 0 & 1
\end{bmatrix}
\end{equation}

\section*{Problem 2}
\section*{Problem 3}
\end{document}